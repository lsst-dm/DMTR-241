% generated from JIRA project LVV
% using template at /usr/share/miniconda/envs/docsteady-env/lib/python3.7/site-packages/docsteady/templates/tpr.latex.jinja2.
% using docsteady version 2.4.1
% Please do not edit -- update information in Jira instead
\documentclass[DM,lsstdraft,STR,toc]{lsstdoc}
\usepackage{geometry}
\usepackage{longtable,booktabs}
\usepackage{enumitem}
\usepackage{arydshln}
\usepackage{attachfile}
\usepackage{array}
\usepackage{dashrule}
\usepackage{pdfpages}

\newcolumntype{L}[1]{>{\raggedright\let\newline\\\arraybackslash\hspace{0pt}}p{#1}}

\input meta.tex

\newcommand{\attachmentsUrl}{https://github.com/\gitorg/\lsstDocType-\lsstDocNum/blob/\gitref/attachments}
\providecommand{\tightlist}{
  \setlength{\itemsep}{0pt}\setlength{\parskip}{0pt}}

\setcounter{tocdepth}{4}

\begin{document}

\def\milestoneName{Network Pre-Verification for Operation Rehearsal \#2}
\def\milestoneId{}
\def\product{Data Management}

\setDocCompact{true}

\title{LVV-P73: Network Pre-Verification for Operation Rehearsal \#2 Test Plan and Report}
\setDocRef{\lsstDocType-\lsstDocNum}
\date{ 2023-08-17 }
\author{ Jeff Kantor }

% Most recent last
\setDocChangeRecord{
\addtohist{}{2023-08-18}{First draft}{Leanne Guy}
\addtohist{}{2024-12-13}{Some executions}{Cristían Silva}
}

\setDocCurator{Leanne Guy}
\setDocUpstreamLocation{\url{https://github.com/lsst-dm/\lsstDocType-\lsstDocNum}}
\setDocUpstreamVersion{\vcsRevision}



\setDocAbstract{
This is the test plan and report for
\textbf{ Network Pre-Verification for Operation Rehearsal \#2},
an LSST milestone pertaining to the Data Management Subsystem.\\
This document is based on content automatically extracted from the Jira test database on \docDate.
The most recent change to the document repository was on \vcsDate.
}


\maketitle

\section{Introduction}
\label{sect:intro}


\subsection{Objectives}
\label{sect:objectives}

 This test campaign will verify that the network infrastructure is ready
for Operation Rehearsal \#2.



\subsection{System Overview}
\label{sect:systemoverview}

 The component of DM subsystem involved in this test campaign is the
\textbf{Base to Archive} network.\\[2\baselineskip]\textbf{Applicable
Documents:}\\[2\baselineskip]\citeds{LSE-61} Data Management System
Requirements\\
\citeds{LDM-732} Rubin Observatory Network Verification Baseline


\subsection{Document Overview}
\label{sect:docoverview}

This document was generated from Jira, obtaining the relevant information from the
\href{https://jira.lsstcorp.org/secure/Tests.jspa\#/testPlan/LVV-P73}{LVV-P73}
~Jira Test Plan and related Test Cycles (
\href{https://jira.lsstcorp.org/secure/Tests.jspa\#/testCycle/LVV-C155}{LVV-C155}
).

Section \ref{sect:intro} provides an overview of the test campaign, the system under test (\product{}),
the applicable documentation, and explains how this document is organized.
Section \ref{sect:testplan} provides additional information about the test plan, like for example the configuration
used for this test or related documentation.
Section \ref{sect:personnel} describes the necessary roles and lists the individuals assigned to them.

Section \ref{sect:overview} provides a summary of the test results, including an overview in Table \ref{table:summary},
an overall assessment statement and suggestions for possible improvements.
Section \ref{sect:detailedtestresults} provides detailed results for each step in each test case.

The current status of test plan \href{https://jira.lsstcorp.org/secure/Tests.jspa\#/testPlan/LVV-P73}{LVV-P73} in Jira is \textbf{ Approved }.

\subsection{References}
\label{sect:references}
\renewcommand{\refname}{}
\bibliography{lsst,refs,books,refs_ads,local}


\newpage
\section{Test Plan Details}
\label{sect:testplan}


\subsection{Data Collection}

  Observing is not required for this test campaign.

\subsection{Verification Environment}
\label{sect:hwconf}
  Base Facility La Serena, Base - Archive Network, Construction Data
Facility at NCSA

  \subsection{Entry Criteria}
  ComCAM data staged in La Serena\\
Base to Archive network functional\\
Construction Data Facility ready to receive data

  \subsection{Exit Criteria}
  Data transferred\\
Network monitoring logs captured.


\subsection{Related Documentation}

Docushare collection where additional relevant documentation can be found:

\begin{itemize}
\item Network monitoring logs
\end{itemize}


\subsection{PMCS Activity}

Primavera milestones related to the test campaign:
NA


\newpage
\section{Personnel}
\label{sect:personnel}

The personnel involved in the test campaign is shown in the following table.

{\small
\begin{longtable}{p{3cm}p{3cm}p{3cm}p{6cm}}
\hline
\multicolumn{2}{r}{T. Plan \href{https://jira.lsstcorp.org/secure/Tests.jspa\#/testPlan/LVV-P73}{LVV-P73} owner:} &
\multicolumn{2}{l}{\textbf{ Jeff Kantor } }\\\hline
\multicolumn{2}{r}{T. Cycle \href{https://jira.lsstcorp.org/secure/Tests.jspa\#/testCycle/LVV-C155}{LVV-C155} owner:} &
\multicolumn{2}{l}{\textbf{
Jeff Kantor }
} \\\hline
\textbf{Test Cases} & \textbf{Assigned to} & \textbf{Executed by} & \textbf{Additional Test Personnel} \\ \hline
\href{https://jira.lsstcorp.org/secure/Tests.jspa#/testCase/LVV-T193}{LVV-T193}
& {\small Jeff Kantor } & {\small  } &
\begin{minipage}[]{6cm}
\smallskip
{\small Josh Hoblitt (Rubin Obs), Renata Frez (FIU/AmLight), Matt Kollross
(NCSA) }
\medskip
\end{minipage}
\\ \hline
\href{https://jira.lsstcorp.org/secure/Tests.jspa#/testCase/LVV-T194}{LVV-T194}
& {\small Jeff Kantor } & {\small  } &
\begin{minipage}[]{6cm}
\smallskip
{\small Josh Hoblitt (Rubin Obs), Renata Frez (FIU/AmLight), Matt Kollross
(NCSA) }
\medskip
\end{minipage}
\\ \hline
\href{https://jira.lsstcorp.org/secure/Tests.jspa#/testCase/LVV-T195}{LVV-T195}
& {\small Jeff Kantor } & {\small  } &
\begin{minipage}[]{6cm}
\smallskip
{\small Josh Hoblitt (Rubin Obs), Renata Frez (FIU/AmLight), Matt Kollross
(NCSA) }
\medskip
\end{minipage}
\\ \hline
\href{https://jira.lsstcorp.org/secure/Tests.jspa#/testCase/LVV-T196}{LVV-T196}
& {\small Jeff Kantor } & {\small  } &
\begin{minipage}[]{6cm}
\smallskip
{\small Josh Hoblitt (Rubin Obs), Renata Frez (FIU/AmLight), Matt Kollross
(NCSA) }
\medskip
\end{minipage}
\\ \hline
\end{longtable}
}

\newpage

\section{Test Campaign Overview}
\label{sect:overview}

\subsection{Summary}
\label{sect:summarytable}

{\small
\begin{longtable}{p{2cm}cp{2.3cm}p{8.6cm}p{2.3cm}}
\toprule
\multicolumn{2}{r}{ T. Plan \href{https://jira.lsstcorp.org/secure/Tests.jspa\#/testPlan/LVV-P73}{LVV-P73}:} &
\multicolumn{2}{p{10.9cm}}{\textbf{ Network Pre-Verification for Operation Rehearsal \#2 }} & Approved \\\hline
\multicolumn{2}{r}{ T. Cycle \href{https://jira.lsstcorp.org/secure/Tests.jspa\#/testCycle/LVV-C155}{LVV-C155}:} &
\multicolumn{2}{p{10.9cm}}{\textbf{ DM Network Testing for Operation Rehearsal \#2 }} & Not Executed \\\hline
\textbf{Test Cases} &  \textbf{Ver.} & \textbf{Status} & \textbf{Comment} & \textbf{Issues} \\\toprule
\href{https://jira.lsstcorp.org/secure/Tests.jspa#/testCase/LVV-T193}{LVV-T193}
&  1
& Not Executed &
\begin{minipage}[]{9cm}
\smallskip

\medskip
\end{minipage}
&   \\\hline
\href{https://jira.lsstcorp.org/secure/Tests.jspa#/testCase/LVV-T194}{LVV-T194}
&  1
& Not Executed &
\begin{minipage}[]{9cm}
\smallskip

\medskip
\end{minipage}
&   \\\hline
\href{https://jira.lsstcorp.org/secure/Tests.jspa#/testCase/LVV-T195}{LVV-T195}
&  1
& Not Executed &
\begin{minipage}[]{9cm}
\smallskip

\medskip
\end{minipage}
&   \\\hline
\href{https://jira.lsstcorp.org/secure/Tests.jspa#/testCase/LVV-T196}{LVV-T196}
&  1
& Not Executed &
\begin{minipage}[]{9cm}
\smallskip

\medskip
\end{minipage}
&   \\\hline
\caption{Test Campaign Summary}
\label{table:summary}
\end{longtable}
}

\subsection{Overall Assessment}
\label{sect:overallassessment}

Not yet available.

\subsection{Recommended Improvements}
\label{sect:recommendations}

Not yet available.

\newpage
\section{Detailed Test Results}
\label{sect:detailedtestresults}

\subsection{Test Cycle LVV-C155 }

Open test cycle {\it \href{https://jira.lsstcorp.org/secure/Tests.jspa#/testrun/LVV-C155}{DM Network Testing for Operation Rehearsal \#2}} in Jira.

Test Cycle name: DM Network Testing for Operation Rehearsal \#2\\
Status: Not Executed

This test cycle includes the list of test cases required to verify the
network infrastructure in the context of this test campaign.

\subsubsection{Software Version/Baseline}
Not provided.

\subsubsection{Configuration}
Not provided.

\subsubsection{Test Cases in LVV-C155 Test Cycle}

\paragraph{ LVV-T193 - Verify implementation of Base to Archive Network }\mbox{}\\

Version \textbf{1}.
Status \textbf{Draft}.
Open  \href{https://jira.lsstcorp.org/secure/Tests.jspa#/testCase/LVV-T193}{\textit{ LVV-T193 } }
test case in Jira.

Verify that the data acquired by a DAQ can be transferred within the
required time, i.e. verify that link is capable of transferring image
for prompt processing in oArchiveMaxTransferTime = 5{[}second{]}, i.e.
at or exceeding rates specified in \citeds{LDM-142}.

\textbf{ Preconditions}:\\
\begin{enumerate}
\tightlist
\item
  Archiver/Forwarders are configured at Base, connected to REUNA DWDM,
  loaded with simulated or pre-cursor data, running on end node
  computers that are the production hardware or equivalent to it.
\item
  Archiver/Forwarder receivers or other capability is on configured at
  LDF, connected to Base - Archive Network, running on end node
  computers that are the production hardware or equivalent to it.
\item
  As-built documentation for all of the above is available.
\end{enumerate}

NOTE: This test will be repeated at increasing data volumes as
additional observatory capabilities (e.g. ComCAM, FullCam) become
available. ~Final verification will be tested at full operational
volume. After the initial test, the corresponding verification elements
will be flagged as ``Requires Monitoring'' such that those requirements
will be closed out as having been verified but will continue to be
monitored throughout commissioning to ensure they do not drop out of
compliance. This will also be monitored for end to end Summit - Data
Facility transfers during Commissioning.

Execution status: {\bf Not Executed }

Final comment:\\


Detailed steps results:

\begin{tabular}{p{2cm}p{14cm}}
\toprule
Step 1 & Step Execution Status: \textbf{ Not Executed } \\ \hline
\end{tabular}
 Description \\
{\footnotesize
Transfer data between base and archive while monitoring the network over
uninterrupted 1 day period (with repeated transfers on normal observing
cadence).

}
\hdashrule[0.5ex]{\textwidth}{1pt}{3mm}
  Test Data \\
 {\footnotesize
LATISS, ComCAM, or FullCAM data.

}
\hdashrule[0.5ex]{\textwidth}{1pt}{3mm}
  Expected Result \\
{\footnotesize
Data transfers occur without significant delay or frequent latency
spikes.

}
\hdashrule[0.5ex]{\textwidth}{1pt}{3mm}
  Actual Result \\
{\footnotesize

}
\begin{tabular}{p{2cm}p{14cm}}
\toprule
Step 2 & Step Execution Status: \textbf{ Not Executed } \\ \hline
\end{tabular}
 Description \\
{\footnotesize
~Analyze the network logs and monitoring system to determine average and
peak latency and packet loss statistics.

}
\hdashrule[0.5ex]{\textwidth}{1pt}{3mm}
  Expected Result \\
{\footnotesize
Data can be transferred within the required time, i.e. verify that link
is capable of transferring image for prompt processing in
oArchiveMaxTransferTime = 5{[}second{]}. Verify transfer of data at or
exceeding rates specified in LDM-142 at least 98\% of the time.

}
\hdashrule[0.5ex]{\textwidth}{1pt}{3mm}
  Actual Result \\
{\footnotesize

}

\paragraph{ LVV-T194 - Verify implementation of Base to Archive Network Availability }\mbox{}\\

Version \textbf{1}.
Status \textbf{Draft}.
Open  \href{https://jira.lsstcorp.org/secure/Tests.jspa#/testCase/LVV-T194}{\textit{ LVV-T194 } }
test case in Jira.

Verify the availability of the Base to Archive Network communications by
demonstrating that it meets or exceeds a mean time between failures,
measured over a 1-yr period of MTBF \textgreater{} baseToArchNetMTBF
(180{[}day{]})

\textbf{ Preconditions}:\\
\begin{enumerate}
\tightlist
\item
  Archiver/Forwarders are configured at Base, connected to REUNA DWDM,
  loaded with simulated or pre-cursor data, running on end node
  computers that are the production hardware or equivalent to it.
\item
  Archiver/Forwarder receivers or other capability is on configured at
  LDF, connected to Base - Archive Network, running on end node
  computers that are the production hardware or equivalent to it.
\item
  At least 6 months of historical monitoring data on this link is
  available.
\item
  As-built documentation for all of the above is available.
\end{enumerate}

NOTE: This test will be repeated at increasing data volumes as
additional observatory capabilities (e.g. ComCAM, FullCam) become
available. Final verification will be tested at full operational volume.
After the initial test, the corresponding verification elements will be
flagged as ``Requires Monitoring'' such that those requirements will be
closed out as having been verified but will continue to be monitored
throughout commissioning to ensure they do not drop out of compliance.
This will also be monitored for end to end Summit - Data Facility
transfers during Commissioning.

Execution status: {\bf Not Executed }

Final comment:\\


Detailed steps results:

\begin{tabular}{p{2cm}p{14cm}}
\toprule
Step 1 & Step Execution Status: \textbf{ Not Executed } \\ \hline
\end{tabular}
 Description \\
{\footnotesize
Transfer data between base and archive over uninterrupted 1 week period.

}
\hdashrule[0.5ex]{\textwidth}{1pt}{3mm}
  Test Data \\
 {\footnotesize
LATISS, ComCAM, or FullCAM data.

}
\hdashrule[0.5ex]{\textwidth}{1pt}{3mm}
  Expected Result \\
{\footnotesize
Data is successfully transferred during the entire week.

}
\hdashrule[0.5ex]{\textwidth}{1pt}{3mm}
  Actual Result \\
{\footnotesize

}
\begin{tabular}{p{2cm}p{14cm}}
\toprule
Step 2 & Step Execution Status: \textbf{ Not Executed } \\ \hline
\end{tabular}
 Description \\
{\footnotesize
Analyze monitoring/performance data, compare to historical data, and
extrapolate to a full year, average and peak throughput and latency.

}
\hdashrule[0.5ex]{\textwidth}{1pt}{3mm}
  Test Data \\
 {\footnotesize
NA

}
\hdashrule[0.5ex]{\textwidth}{1pt}{3mm}
  Expected Result \\
{\footnotesize
Extrapolated network availability meets baseToArchNetMTBF =
180{[}day{]}. ~Note that this is for complete loss of transfer service
(all paths), not a single path failure with successful fail-over.

}
\hdashrule[0.5ex]{\textwidth}{1pt}{3mm}
  Actual Result \\
{\footnotesize

}

\paragraph{ LVV-T195 - Verify implementation of Base to Archive Network Reliability }\mbox{}\\

Version \textbf{1}.
Status \textbf{Draft}.
Open  \href{https://jira.lsstcorp.org/secure/Tests.jspa#/testCase/LVV-T195}{\textit{ LVV-T195 } }
test case in Jira.

Verify Base to Archive Network Reliability by demonstrating that the
network can recover from outages within baseToArchNetMTTR =
48{[}hour{]}.

\textbf{ Preconditions}:\\
\begin{enumerate}
\tightlist
\item
  Archiver/Forwarders are configured at Base, connected to REUNA DWDM,
  loaded with simulated or pre-cursor data, running on end node
  computers that are the production hardware or equivalent to it.
\item
  Archiver/Forwarder receivers or other capability is on configured at
  LDF, connected to Base - Archive Network, running on end node
  computers that are the production hardware or equivalent to it.
\item
  At least 6 months of monitoring data for this link is available.
\item
  As-built documentation for all of the above is available.
\end{enumerate}

NOTE: This test will be repeated at increasing data volumes as
additional observatory capabilities (e.g. ComCAM, FullCam) become
available. Final verification will be tested at full operational volume.
After the initial test, the corresponding verification elements will be
flagged as ``Requires Monitoring'' such that those requirements will be
closed out as having been verified but will continue to be monitored
throughout commissioning to ensure they do not drop out of compliance.
This will also be monitored for end to end Summit - Data Facility
transfers during Commissioning.\\[2\baselineskip]

Execution status: {\bf Not Executed }

Final comment:\\


Detailed steps results:

\begin{tabular}{p{2cm}p{14cm}}
\toprule
Step 1 & Step Execution Status: \textbf{ Not Executed } \\ \hline
\end{tabular}
 Description \\
{\footnotesize
Disconnect primary fiber on base side of Base - ~Archive network.

}
\hdashrule[0.5ex]{\textwidth}{1pt}{3mm}
  Test Data \\
 {\footnotesize
LATISS, ComCAM, or FullCAM data.

}
\hdashrule[0.5ex]{\textwidth}{1pt}{3mm}
  Expected Result \\
{\footnotesize
Network fails over to secondary path.

}
\hdashrule[0.5ex]{\textwidth}{1pt}{3mm}
  Actual Result \\
{\footnotesize

}
\begin{tabular}{p{2cm}p{14cm}}
\toprule
Step 2 & Step Execution Status: \textbf{ Not Executed } \\ \hline
\end{tabular}
 Description \\
{\footnotesize
Simulate diagnosis and repair by elapsed time.\\[2\baselineskip]

}
\hdashrule[0.5ex]{\textwidth}{1pt}{3mm}
  Test Data \\
 {\footnotesize
NA

}
\hdashrule[0.5ex]{\textwidth}{1pt}{3mm}
  Expected Result \\
{\footnotesize
Wall clock advances by 48 hours. ~Data is successfully transferred over
secondary path.

}
\hdashrule[0.5ex]{\textwidth}{1pt}{3mm}
  Actual Result \\
{\footnotesize

}
\begin{tabular}{p{2cm}p{14cm}}
\toprule
Step 3 & Step Execution Status: \textbf{ Not Executed } \\ \hline
\end{tabular}
 Description \\
{\footnotesize
Reconnect primary fiber on base side of Base - Archive network.

}
\hdashrule[0.5ex]{\textwidth}{1pt}{3mm}
  Test Data \\
 {\footnotesize
NA

}
\hdashrule[0.5ex]{\textwidth}{1pt}{3mm}
  Expected Result \\
{\footnotesize
Network recovers to primary path.~

}
\hdashrule[0.5ex]{\textwidth}{1pt}{3mm}
  Actual Result \\
{\footnotesize

}
\begin{tabular}{p{2cm}p{14cm}}
\toprule
Step 4 & Step Execution Status: \textbf{ Not Executed } \\ \hline
\end{tabular}
 Description \\
{\footnotesize
Analyze fail-over and recovery times. ~Compare to historical data and
extrapolate to MTTR.

}
\hdashrule[0.5ex]{\textwidth}{1pt}{3mm}
  Expected Result \\
{\footnotesize
Verify recovery can occur within baseToArchNetMTTR = 48{[}hour{]}.
Demonstrate reconnection and recovery to transfer of data at or
exceeding rates specified in LDM-142.

}
\hdashrule[0.5ex]{\textwidth}{1pt}{3mm}
  Actual Result \\
{\footnotesize

}

\paragraph{ LVV-T196 - Verify implementation of Base to Archive Network Secondary Link }\mbox{}\\

Version \textbf{1}.
Status \textbf{Draft}.
Open  \href{https://jira.lsstcorp.org/secure/Tests.jspa#/testCase/LVV-T196}{\textit{ LVV-T196 } }
test case in Jira.

Verify Base to Archive Network Secondary Link failover and capacity, and
subsequent recovery primary. Demonstrate the use of the secondary path
in ``catch-up'' mode.

\textbf{ Preconditions}:\\
\begin{enumerate}
\tightlist
\item
  Archiver/Forwarders are configured at Base, connected to REUNA DWDM,
  loaded with simulated or pre-cursor data, running on end node
  computers that are the production hardware or equivalent to it.
\item
  Archiver/Forwarder receivers or other capability is on configured at
  LDF, connected to Base - Archive Network, running on end node
  computers that are the production hardware or equivalent to it.
\item
  As-built documentation for all of the above is available.
\end{enumerate}

NOTE: This test will be repeated at increasing data volumes as
additional observatory capabilities (e.g. ComCAM, FullCam) become
available. Final verification will be tested at full operational volume.
After the initial test, the corresponding verification elements will be
flagged as ``Requires Monitoring'' such that those requirements will be
closed out as having been verified but will continue to be monitored
throughout commissioning to ensure they do not drop out of compliance.
This will also be monitored for end to end Summit - Data Facility
transfers during Commissioning.\\[2\baselineskip]

Execution status: {\bf Not Executed }

Final comment:\\


Detailed steps results:

\begin{tabular}{p{2cm}p{14cm}}
\toprule
Step 1 & Step Execution Status: \textbf{ Not Executed } \\ \hline
\end{tabular}
 Description \\
{\footnotesize
Transfer data between base and archive on primary links over
uninterrupted 1 day period.

}
\hdashrule[0.5ex]{\textwidth}{1pt}{3mm}
  Test Data \\
 {\footnotesize
LATISS, ComCAM, or FullCAM data.

}
\hdashrule[0.5ex]{\textwidth}{1pt}{3mm}
  Expected Result \\
{\footnotesize
Data is successfully transferred over primary link at or exceeding rates
specified in LDM-142 throughout period.

}
\hdashrule[0.5ex]{\textwidth}{1pt}{3mm}
  Actual Result \\
{\footnotesize

}
\begin{tabular}{p{2cm}p{14cm}}
\toprule
Step 2 & Step Execution Status: \textbf{ Not Executed } \\ \hline
\end{tabular}
 Description \\
{\footnotesize
Simulate outage by disconnecting fiber on primary fiber on Base side of
Base - Archive Network.

}
\hdashrule[0.5ex]{\textwidth}{1pt}{3mm}
  Test Data \\
 {\footnotesize
NA

}
\hdashrule[0.5ex]{\textwidth}{1pt}{3mm}
  Expected Result \\
{\footnotesize
Network fails over to secondary links in \textless{}=60s

}
\hdashrule[0.5ex]{\textwidth}{1pt}{3mm}
  Actual Result \\
{\footnotesize

}
\begin{tabular}{p{2cm}p{14cm}}
\toprule
Step 3 & Step Execution Status: \textbf{ Not Executed } \\ \hline
\end{tabular}
 Description \\
{\footnotesize
Transfer data between base and archive over secondary equipment
uninterrupted 1 day period.

}
\hdashrule[0.5ex]{\textwidth}{1pt}{3mm}
  Test Data \\
 {\footnotesize
LATISS, ComCAM, or FullCAM data.

}
\hdashrule[0.5ex]{\textwidth}{1pt}{3mm}
  Expected Result \\
{\footnotesize
Data is successfully transferred over secondary link ~at or exceeding
rates specified in LDM-142 throughout period.

}
\hdashrule[0.5ex]{\textwidth}{1pt}{3mm}
  Actual Result \\
{\footnotesize

}
\begin{tabular}{p{2cm}p{14cm}}
\toprule
Step 4 & Step Execution Status: \textbf{ Not Executed } \\ \hline
\end{tabular}
 Description \\
{\footnotesize
Restore connection on primary link by reconnecting
fiber.\\[2\baselineskip]

}
\hdashrule[0.5ex]{\textwidth}{1pt}{3mm}
  Test Data \\
 {\footnotesize
NA

}
\hdashrule[0.5ex]{\textwidth}{1pt}{3mm}
  Expected Result \\
{\footnotesize
Network recovers to primary.

}
\hdashrule[0.5ex]{\textwidth}{1pt}{3mm}
  Actual Result \\
{\footnotesize

}
\begin{tabular}{p{2cm}p{14cm}}
\toprule
Step 5 & Step Execution Status: \textbf{ Not Executed } \\ \hline
\end{tabular}
 Description \\
{\footnotesize
Demonstrate use of secondary in catch-up mode.

}
\hdashrule[0.5ex]{\textwidth}{1pt}{3mm}
  Test Data \\
 {\footnotesize
DAQ buffer full of images and associated metadata.

}
\hdashrule[0.5ex]{\textwidth}{1pt}{3mm}
  Expected Result \\
{\footnotesize
Images from DAQ buffer and associated metadata are retrievable over
secondary path while current observing data is being transferred over
primary path.

}
\hdashrule[0.5ex]{\textwidth}{1pt}{3mm}
  Actual Result \\
{\footnotesize

}




% This appendix is put in as part of the template. You may edit and add to it.
% It is not overwritten by Docsteady.

\newpage
\appendix
\section{Documentation}
The verification process is defined in \citeds{LSE-160}.
The use of Docsteady to format Jira information in various test and planing documents is
described in \citeds{DMTN-140} and practical commands are given in \citeds{DMTN-178}.

\section{Acronyms used in this document}\label{sec:acronyms}
\input{acronyms.tex}

\newpage

% Uncomment this if Docsteady makes you additional appendix
%% generated from JIRA project LVV
% using template at /usr/local/lib/python3.7/site-packages/docsteady/templates/dm-tpr-appendix.latex.jinja2.
% using docsteady version 1.2rc24
% Please do not edit -- update information in Jira instead

\section{Traceability}

\begin{longtable}{p{3cm}p{3cm}L{9cm}}
\hline
\textbf{Test Case} & \textbf{VE Key} & \textbf{VE Summary} \\ \hline
\href{https://jira.lsstcorp.org/secure/Tests.jspa#/testCase/LVV-T193}{LVV-T193} &
  \href{https://jira.lsstcorp.org/browse/LVV-81}{LVV-81}
  & DMS-REQ-0180-V-01: Base to Archive Network
 \\ \cdashline{2-3}
\hline
\href{https://jira.lsstcorp.org/secure/Tests.jspa#/testCase/LVV-T194}{LVV-T194} &
  \href{https://jira.lsstcorp.org/browse/LVV-82}{LVV-82}
  & DMS-REQ-0181-V-01: Base to Archive Network Availability
 \\ \cdashline{2-3}
\hline
\href{https://jira.lsstcorp.org/secure/Tests.jspa#/testCase/LVV-T195}{LVV-T195} &
  \href{https://jira.lsstcorp.org/browse/LVV-83}{LVV-83}
  & DMS-REQ-0182-V-01: Base to Archive Network Reliability
 \\ \cdashline{2-3}
\hline
\href{https://jira.lsstcorp.org/secure/Tests.jspa#/testCase/LVV-T196}{LVV-T196} &
  \href{https://jira.lsstcorp.org/browse/LVV-84}{LVV-84}
  & DMS-REQ-0183-V-01: Base to Archive Network Secondary Link
 \\ \cdashline{2-3}
\hline
\end{longtable}


\end{document}
