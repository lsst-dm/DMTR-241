% generated from JIRA project LVV
% using template at /usr/share/miniconda/envs/docsteady-env/lib/python3.7/site-packages/docsteady/templates/tpnoresult.latex.jinja2.
% using docsteady version 2.4.1
% Please do not edit -- update information in Jira instead
\documentclass[DM,lsstdraft,STR,toc]{lsstdoc}
\usepackage{geometry}
\usepackage{longtable,booktabs}
\usepackage{enumitem}
\usepackage{arydshln}
\usepackage{attachfile}
\usepackage{array}
\usepackage{dashrule}

\newcolumntype{L}[1]{>{\raggedright\let\newline\\\arraybackslash\hspace{0pt}}p{#1}}

\input meta.tex

\newcommand{\attachmentsUrl}{https://github.com/\gitorg/\lsstDocType-\lsstDocNum/blob/\gitref/attachments}
\providecommand{\tightlist}{
  \setlength{\itemsep}{0pt}\setlength{\parskip}{0pt}}

\setcounter{tocdepth}{4}

\begin{document}

\def\milestoneName{Network Acceptance Test Campaign 1}
\def\milestoneId{}
\def\product{Data Management}

\setDocCompact{true}

\title{LVV-P73: Network Acceptance Test Campaign 1 Test Plan }
\setDocRef{\lsstDocType-\lsstDocNum}
\date{ 2023-08-18 }
\author{ Leanne Guy }

% Most recent last
\setDocChangeRecord{
\addtohist{}{2023-08-18}{First draft}{Leanne Guy}
\addtohist{}{2024-12-13}{Some executions}{Cristían Silva}
}

\setDocCurator{Leanne Guy}
\setDocUpstreamLocation{\url{https://github.com/lsst-dm/\lsstDocType-\lsstDocNum}}
\setDocUpstreamVersion{\vcsRevision}



\setDocAbstract{
This is the test plan for
\textbf{ Network Acceptance Test Campaign 1},
an LSST milestone pertaining to the Data Management Subsystem.\\
This document is based on content automatically extracted from the Jira test database on \docDate.
The most recent change to the document repository was on \vcsDate.
}


\maketitle

\section{Introduction}
\label{sect:intro}


\subsection{Objectives}
\label{sect:objectives}

 This test campaign will verify the network infrastructure without
encryption



\subsection{System Overview}
\label{sect:systemoverview}

 The component of DM subsystem involved in this test campaign is the
\textbf{Base to Archive} network.\\[2\baselineskip]\textbf{Applicable
Documents:}\\[2\baselineskip]\citeds{LSE-61} Data Management System
Requirements\\
\citeds{LDM-503} Data Management Test Plan\\
\citeds{LDM-639} Data Management Acceptance Test Specification\\
\citeds{LDM-732} Rubin Observatory Network Verification Baseline


\subsection{Document Overview}
\label{sect:docoverview}

This document was generated from Jira, obtaining the relevant information from the
\href{https://jira.lsstcorp.org/secure/Tests.jspa\#/testPlan/LVV-P73}{LVV-P73}
~Jira Test Plan and related Test Cycles (
\href{https://jira.lsstcorp.org/secure/Tests.jspa\#/testCycle/LVV-C155}{LVV-C155}
).

Section \ref{sect:intro} provides an overview of the test campaign, the system under test (\product{}),
the applicable documentation, and explains how this document is organized.
Section \ref{sect:testplan} provides additional information about the test plan, like for example the configuration
used for this test or related documentation.
Section \ref{sect:personnel} describes the necessary roles and lists the individuals assigned to them.

Section \ref{sect:overview} provides a summary of the test results, including an overview in Table \ref{table:summary},
an overall assessment statement and suggestions for possible improvements.
Section \ref{sect:detailedtestresults} provides detailed results for each step in each test case.

The current status of test plan \href{https://jira.lsstcorp.org/secure/Tests.jspa\#/testPlan/LVV-P73}{LVV-P73} in Jira is \textbf{ Draft }.

\subsection{References}
\label{sect:references}
\renewcommand{\refname}{}
\bibliography{lsst,refs,books,refs_ads,local}


\newpage
\section{Test Plan Details}
\label{sect:testplan}


\subsection{Data Collection}

  Observing is not required for this test campaign.

\subsection{Verification Environment}
\label{sect:hwconf}
  Base Facility La Serena, Base - Archive Network, ~Data Facility at SLAC

  \subsection{Entry Criteria}
  Base to Archive network functional\\
Construction Data Facility ready to receive data

  \subsection{Exit Criteria}
  Data transferred\\
Network monitoring logs captured.


\subsection{Related Documentation}

Docushare collection where additional relevant documentation can be found:

\begin{itemize}
\item Network monitoring logs
\end{itemize}



\subsection{PMCS Activity}

Primavera milestones related to the test campaign:
NA


\newpage
\section{Personnel}
\label{sect:personnel}

The personnel involved in the test campaign is shown in the following table.

{\small
\begin{longtable}{p{3cm}p{3cm}p{3cm}p{6cm}}
\hline
\multicolumn{2}{r}{T. Plan \href{https://jira.lsstcorp.org/secure/Tests.jspa\#/testPlan/LVV-P73}{LVV-P73} owner:} &
\multicolumn{2}{l}{\textbf{ Leanne Guy } }\\\hline
\multicolumn{2}{r}{T. Cycle \href{https://jira.lsstcorp.org/secure/Tests.jspa\#/testCycle/LVV-C155}{LVV-C155} owner:} &
\multicolumn{2}{l}{\textbf{
Leanne Guy }
} \\\hline
\textbf{Test Cases} & \textbf{Assigned to} & \textbf{Executed by} & \textbf{Additional Test Personnel} \\ \hline
\href{https://jira.lsstcorp.org/secure/Tests.jspa#/testCase/LVV-T203}{LVV-T203}
& {\small Kian-Tat Lim } & {\small  } &
\begin{minipage}[]{6cm}
\smallskip
{\small Ron Lambert (LSST), Albert Astudillo (REUNA), Jeronimo Bezerra
(FIU/AmLight), Matt Kollross (NCSA) }
\medskip
\end{minipage}
\\ \hline
\href{https://jira.lsstcorp.org/secure/Tests.jspa#/testCase/LVV-T181}{LVV-T181}
& {\small Jeff Kantor } & {\small  } &
\begin{minipage}[]{6cm}
\smallskip
{\small Heinrich Reinking (LSST), another LSST DM Person at Summit,
Headquarters, or LDF }
\medskip
\end{minipage}
\\ \hline
\href{https://jira.lsstcorp.org/secure/Tests.jspa#/testCase/LVV-T1612}{LVV-T1612}
& {\small Jeff Kantor } & {\small  } &
\begin{minipage}[]{6cm}
\smallskip
{\small Ron Lambert (LSST), Greg Thayer (SLAC) }
\medskip
\end{minipage}
\\ \hline
\href{https://jira.lsstcorp.org/secure/Tests.jspa#/testCase/LVV-T189}{LVV-T189}
& {\small Robert Gruendl [X] } & {\small  } &
\begin{minipage}[]{6cm}
\smallskip
{\small  }
\medskip
\end{minipage}
\\ \hline
\href{https://jira.lsstcorp.org/secure/Tests.jspa#/testCase/LVV-T192}{LVV-T192}
& {\small Jeff Kantor } & {\small  } &
\begin{minipage}[]{6cm}
\smallskip
{\small Heinrich Reinking (LSST) }
\medskip
\end{minipage}
\\ \hline
\href{https://jira.lsstcorp.org/secure/Tests.jspa#/testCase/LVV-T185}{LVV-T185}
& {\small Jeff Kantor } & {\small  } &
\begin{minipage}[]{6cm}
\smallskip
{\small Ron Lambert (LSST) }
\medskip
\end{minipage}
\\ \hline
\href{https://jira.lsstcorp.org/secure/Tests.jspa#/testCase/LVV-T188}{LVV-T188}
& {\small Jeff Kantor } & {\small  } &
\begin{minipage}[]{6cm}
\smallskip
{\small Jeff Kantor (LSST) }
\medskip
\end{minipage}
\\ \hline
\href{https://jira.lsstcorp.org/secure/Tests.jspa#/testCase/LVV-T186}{LVV-T186}
& {\small Jeff Kantor } & {\small  } &
\begin{minipage}[]{6cm}
\smallskip
{\small Ron Lambert (LSST), Guido Maulen (LSST) }
\medskip
\end{minipage}
\\ \hline
\href{https://jira.lsstcorp.org/secure/Tests.jspa#/testCase/LVV-T187}{LVV-T187}
& {\small Jeff Kantor } & {\small  } &
\begin{minipage}[]{6cm}
\smallskip
{\small Ron Lambert (LSST) }
\medskip
\end{minipage}
\\ \hline
\href{https://jira.lsstcorp.org/secure/Tests.jspa#/testCase/LVV-T193}{LVV-T193}
& {\small Jeff Kantor } & {\small  } &
\begin{minipage}[]{6cm}
\smallskip
{\small Josh Hoblitt (Rubin Obs), Renata Frez (FIU/AmLight), Matt Kollross
(NCSA) }
\medskip
\end{minipage}
\\ \hline
\href{https://jira.lsstcorp.org/secure/Tests.jspa#/testCase/LVV-T194}{LVV-T194}
& {\small Jeff Kantor } & {\small  } &
\begin{minipage}[]{6cm}
\smallskip
{\small Josh Hoblitt (Rubin Obs), Renata Frez (FIU/AmLight), Matt Kollross
(NCSA) }
\medskip
\end{minipage}
\\ \hline
\href{https://jira.lsstcorp.org/secure/Tests.jspa#/testCase/LVV-T195}{LVV-T195}
& {\small Jeff Kantor } & {\small  } &
\begin{minipage}[]{6cm}
\smallskip
{\small Josh Hoblitt (Rubin Obs), Renata Frez (FIU/AmLight), Matt Kollross
(NCSA) }
\medskip
\end{minipage}
\\ \hline
\href{https://jira.lsstcorp.org/secure/Tests.jspa#/testCase/LVV-T196}{LVV-T196}
& {\small Jeff Kantor } & {\small  } &
\begin{minipage}[]{6cm}
\smallskip
{\small Josh Hoblitt (Rubin Obs), Renata Frez (FIU/AmLight), Matt Kollross
(NCSA) }
\medskip
\end{minipage}
\\ \hline
\end{longtable}
}

\newpage

\section{Test Campaign Overview}
\label{sect:overview}

\subsection{Summary}
\label{sect:summarytable}

{\small
\begin{longtable}{p{2cm}cp{2.3cm}p{8.6cm}p{2.3cm}}
\toprule
\multicolumn{2}{r}{ T. Plan \href{https://jira.lsstcorp.org/secure/Tests.jspa\#/testPlan/LVV-P73}{LVV-P73}:} &
\multicolumn{2}{p{10.9cm}}{\textbf{ Network Acceptance Test Campaign 1 }} & Draft \\\hline
\multicolumn{2}{r}{ T. Cycle \href{https://jira.lsstcorp.org/secure/Tests.jspa\#/testCycle/LVV-C155}{LVV-C155}:} &
\multicolumn{2}{p{10.9cm}}{\textbf{ Network Acceptance Test Campaign 1 }} & Not Executed \\\hline
\textbf{Test Cases} &  \textbf{Ver.}  \\\toprule
\href{https://jira.lsstcorp.org/secure/Tests.jspa#/testCase/LVV-T203}{LVV-T203}
&  1
\\
\href{https://jira.lsstcorp.org/secure/Tests.jspa#/testCase/LVV-T181}{LVV-T181}
&  1
\\
\href{https://jira.lsstcorp.org/secure/Tests.jspa#/testCase/LVV-T1612}{LVV-T1612}
&  1
\\
\href{https://jira.lsstcorp.org/secure/Tests.jspa#/testCase/LVV-T189}{LVV-T189}
&  1
\\
\href{https://jira.lsstcorp.org/secure/Tests.jspa#/testCase/LVV-T192}{LVV-T192}
&  1
\\
\href{https://jira.lsstcorp.org/secure/Tests.jspa#/testCase/LVV-T185}{LVV-T185}
&  1
\\
\href{https://jira.lsstcorp.org/secure/Tests.jspa#/testCase/LVV-T188}{LVV-T188}
&  1
\\
\href{https://jira.lsstcorp.org/secure/Tests.jspa#/testCase/LVV-T186}{LVV-T186}
&  1
\\
\href{https://jira.lsstcorp.org/secure/Tests.jspa#/testCase/LVV-T187}{LVV-T187}
&  1
\\
\href{https://jira.lsstcorp.org/secure/Tests.jspa#/testCase/LVV-T193}{LVV-T193}
&  1
\\
\href{https://jira.lsstcorp.org/secure/Tests.jspa#/testCase/LVV-T194}{LVV-T194}
&  1
\\
\href{https://jira.lsstcorp.org/secure/Tests.jspa#/testCase/LVV-T195}{LVV-T195}
&  1
\\
\href{https://jira.lsstcorp.org/secure/Tests.jspa#/testCase/LVV-T196}{LVV-T196}
&  1
\\
\\\hline
\caption{Test Campaign Summary}
\label{table:summary}
\end{longtable}
}

\subsection{Overall Assessment}
\label{sect:overallassessment}

Not yet available.

\subsection{Recommended Improvements}
\label{sect:recommendations}

\newpage
\section{Detailed Tests}
\label{sect:detailedtests}

\subsection{Test Cycle LVV-C155 }

Open test cycle {\it \href{https://jira.lsstcorp.org/secure/Tests.jspa#/testrun/LVV-C155}{Network Acceptance Test Campaign 1}} in Jira.

Test Cycle name: Network Acceptance Test Campaign 1\\
Status: Not Executed

This test cycle includes the list of test cases required to verify the
network infrastructure without encryption

\subsubsection{Software Version/Baseline}
Not provided.

\subsubsection{Configuration}
Not provided.

\subsubsection{Test Cases in LVV-C155 Test Cycle}

\paragraph{ LVV-T203 - Verify implementation of Archive to Data Access Center Network Secondary
Link }\mbox{}\\

Version \textbf{1}.
Open  \href{https://jira.lsstcorp.org/secure/Tests.jspa#/testCase/LVV-T203}{\textit{ LVV-T203 } }
test case in Jira.

Verify the Archive to Data Access Center Network via Secondary Link by
simulating a failure on the primary path and capacity on the secondary
path.

\textbf{ Preconditions}:\\
\begin{enumerate}
\tightlist
\item
  Data is staged in LDF and data transfer capabilities to US DAC and
  Chilean DAC are in place, running on end node computers that are the
  production hardware or equivalent to it.
\item
  As-built documentation for all of the above is available.
\end{enumerate}

NOTE: This test will be repeated at increasing data volumes as
additional observatory capabilities (e.g. ComCAM, FullCam) become
available. Final verification will be tested at full operational volume.
After the initial test, the corresponding verification elements will be
flagged as ``Requires Monitoring'' such that those requirements will be
closed out as having been verified but will continue to be monitored
throughout commissioning to ensure they do not drop out of compliance.
This will also be monitored for end to end Summit - Data Facility
transfers during Commissioning.\\[2\baselineskip]

Final comment:\\


Detailed steps :

\begin{tabular}{p{2cm}}
\toprule
Step 1  \\ \hline
\end{tabular}
 Description \\
{\footnotesize
Transfer data between Archive and DACs on primary path over
uninterrupted 1 week period.

}
\hdashrule[0.5ex]{\textwidth}{1pt}{3mm}
  Test Data \\
 {\footnotesize
Data Release or other petabyte-scale test data set.

}
\hdashrule[0.5ex]{\textwidth}{1pt}{3mm}
  Expected Result \\
{\footnotesize
Data transfers without failures or extended latency spikes, at or
exceeding rates specified in LDM-142 throughout fail-over period.

}

\begin{tabular}{p{2cm}}
\toprule
Step 2  \\ \hline
\end{tabular}
 Description \\
{\footnotesize
Simulate outage on primary path by disconnecting fiber on primary on
Archive side of Archive - DACs networks.

}
\hdashrule[0.5ex]{\textwidth}{1pt}{3mm}
  Test Data \\
 {\footnotesize
NA

}
\hdashrule[0.5ex]{\textwidth}{1pt}{3mm}
  Expected Result \\
{\footnotesize
Network fails over to secondary links in \textless{}=
60s.\\[2\baselineskip]

}

\begin{tabular}{p{2cm}}
\toprule
Step 3  \\ \hline
\end{tabular}
 Description \\
{\footnotesize
Transfer data between base and archive over secondary equipment
uninterrupted 1 day period.

}
\hdashrule[0.5ex]{\textwidth}{1pt}{3mm}
  Test Data \\
 {\footnotesize
Data Release or other petabyte-scale test data set.

}
\hdashrule[0.5ex]{\textwidth}{1pt}{3mm}
  Expected Result \\
{\footnotesize
Data transfers without failures or extended latency spikes, ~at or
exceeding rates specified in LDM-142 throughout fail-over period.

}

\begin{tabular}{p{2cm}}
\toprule
Step 4  \\ \hline
\end{tabular}
 Description \\
{\footnotesize
Restore connection on primary link (reconnect fiber).

}
\hdashrule[0.5ex]{\textwidth}{1pt}{3mm}
  Test Data \\
 {\footnotesize
NA

}
\hdashrule[0.5ex]{\textwidth}{1pt}{3mm}
  Expected Result \\
{\footnotesize
Network recovers to primary in \textless{}= 60s.

}

\paragraph{ LVV-T181 - Verify Base Voice Over IP (VOIP) }\mbox{}\\

Version \textbf{1}.
Open  \href{https://jira.lsstcorp.org/secure/Tests.jspa#/testCase/LVV-T181}{\textit{ LVV-T181 } }
test case in Jira.

Verify as-built VOIP at the Base Facility is operational and performs as
expected (i.e. sufficient number of extensions allocated properly, no
frequent drop-outs, no frequent jaggies on video, etc.) on both voice
calls and videoconferening.

\textbf{ Preconditions}:\\
\begin{enumerate}
\tightlist
\item
  Base VOIP is installed/configured and Test Personnel have phone sets.
  ~Base Videoconference system is installed/configured. ~Summit,
  Headquarters, and/or LDF Videconference system is
  installed/configured.
\item
  As-built documentation for all of the above is available.
\end{enumerate}

Final comment:\\


Detailed steps :

\begin{tabular}{p{2cm}}
\toprule
Step 1  \\ \hline
\end{tabular}
 Description \\
{\footnotesize
Test voice calls over VOIP system from Base Facility to locations in
~Base and to other Rubin Observatory facilities.

}
\hdashrule[0.5ex]{\textwidth}{1pt}{3mm}
  Expected Result \\
{\footnotesize
As-built VOIP at the Base Facility is operational and performs as
expected (i.e. sufficient number of extensions allocated properly, no
frequent drop-outs, etc.).

}

\begin{tabular}{p{2cm}}
\toprule
Step 2  \\ \hline
\end{tabular}
 Description \\
{\footnotesize
Test video conferences over ~system from Base Facility to locations in
Base and to other Rubin Observatory facilities.

}
\hdashrule[0.5ex]{\textwidth}{1pt}{3mm}
  Expected Result \\
{\footnotesize
Verify (a) plannned and (b) as-built VOIP at the Base Facility is
operational and performs as expected (i.e. no frequent drop-outs, no
frequent audio glitches, no frequent jaggies on video, etc.).

}

\paragraph{ LVV-T1612 - Verify Summit - Base Network Integration (System Level) }\mbox{}\\

Version \textbf{1}.
Open  \href{https://jira.lsstcorp.org/secure/Tests.jspa#/testCase/LVV-T1612}{\textit{ LVV-T1612 } }
test case in Jira.

Verify ISO Layer 3 full (22 x 10 Gbps ethernet ports on DAQ side with
test data from DAQ test stand, AURA, Camera DAQ team do test).
Demonstrate transfer of data at or exceeding rates specified in \citeds{LDM-142}.

\textbf{ Preconditions}:\\
\begin{enumerate}
\tightlist
\item
  PMCS DMTC-7400-2400 COMPLETE
\item
  \href{https://jira.lsstcorp.org/secure/Tests.jspa\#/testCase/1401}{LVV-T1168}
  Passed
\item
  EITHER: Full Camera DAQ installed on summit and loaded with data OR:
  high-quality DAQ application-level simulators that match the form,
  volume, file paths, compressibility, and cadence of the expected
  instrument data, running on end node computers that are the production
  hardware or equivalent to it. Scientific validity of the data content
  is not essential.
\item
  Archiver/forwarders installed at Base running on end node computers
  that are the production hardware or equivalent to it.
\item
  As-built documentation for all of the above is available.
\end{enumerate}

NOTE: This test will be repeated at increasing data volumes as
additional observatory capabilities (e.g. ComCAM, FullCam) become
available. Final verification will be tested at full operational
volume.After the initial test, the corresponding verification elements
will be flagged as ``Requires Monitoring'' such that those requirements
will be closed out as having been verified but will continue to be
monitored throughout commissioning to ensure they do not drop out of
compliance. This will also be monitored for end to end Summit - Data
Facility transfers during Commissioning.

Final comment:\\


Detailed steps :

\begin{tabular}{p{2cm}}
\toprule
Step 1  \\ \hline
\end{tabular}
 Description \\
{\footnotesize
Verify Pre-conditions are satisfied.

}
\hdashrule[0.5ex]{\textwidth}{1pt}{3mm}
  Test Data \\
 {\footnotesize
NA

}
\hdashrule[0.5ex]{\textwidth}{1pt}{3mm}
  Expected Result \\
{\footnotesize
Pre-conditions are satisfied.

}

\begin{tabular}{p{2cm}}
\toprule
Step 2  \\ \hline
\end{tabular}
 Description \\
{\footnotesize
Transfer data between summit and base over uninterrupted 1 day period.
~Monitor transfer of data at or exceeding rates specified in LDM-142.

}
\hdashrule[0.5ex]{\textwidth}{1pt}{3mm}
  Test Data \\
 {\footnotesize
DAQ pre-loaded data

}
\hdashrule[0.5ex]{\textwidth}{1pt}{3mm}
  Expected Result \\
{\footnotesize
Data transfers at or exceeding rates specified in LDM-142.

}

\paragraph{ LVV-T189 - Verify implementation of Base Facility Infrastructure }\mbox{}\\

Version \textbf{1}.
Open  \href{https://jira.lsstcorp.org/secure/Tests.jspa#/testCase/LVV-T189}{\textit{ LVV-T189 } }
test case in Jira.

Verify that the (a) planned infrastructure and (b) as-built
infrastructure for the Base Facility satisfies the needs for data
transfer and buffering, a copy of the Archive Facility, and support for
Commissioning.

\textbf{ Preconditions}:\\


Final comment:\\


Detailed steps :

\begin{tabular}{p{2cm}}
\toprule
Step 1  \\ \hline
\end{tabular}
 Description \\
{\footnotesize
Analyze design and sizing model

}
\hdashrule[0.5ex]{\textwidth}{1pt}{3mm}
  Expected Result \\
{\footnotesize

}

\paragraph{ LVV-T192 - Verify implementation of Base Wireless LAN (WiFi) }\mbox{}\\

Version \textbf{1}.
Open  \href{https://jira.lsstcorp.org/secure/Tests.jspa#/testCase/LVV-T192}{\textit{ LVV-T192 } }
test case in Jira.

Verify as-built wireless network at the Base Facility supports
minBaseWiFi bandwidth (1000 Mbs).

\textbf{ Preconditions}:\\
\begin{enumerate}
\tightlist
\item
  Base Wireless LAN is installed/configured and Test Personnel have
  accounts for email, internet access.
\item
  As-built documentation for all of the above is available.
\end{enumerate}

Final comment:\\


Detailed steps :

\begin{tabular}{p{2cm}}
\toprule
Step 1  \\ \hline
\end{tabular}
 Description \\
{\footnotesize
Test internet web browsing and file download, email at summit and base
over wireless.

}
\hdashrule[0.5ex]{\textwidth}{1pt}{3mm}
  Test Data \\
 {\footnotesize
NA

}
\hdashrule[0.5ex]{\textwidth}{1pt}{3mm}
  Expected Result \\
{\footnotesize
Verify as-built wireless network at the Base Facility supports
minBaseWiFi bandwidth (1000 Mbs). Verify wireless signal strength meets
or exceeds typical, and average and peak bandwidths meet or exceed
minBaseWiFI bandwidth.

}

\paragraph{ LVV-T185 - Verify implementation of Summit to Base Network Availability }\mbox{}\\

Version \textbf{1}.
Open  \href{https://jira.lsstcorp.org/secure/Tests.jspa#/testCase/LVV-T185}{\textit{ LVV-T185 } }
test case in Jira.

Verify the availability of Summit to Base Network by demonstrating that
the mean time between failures is less than summToBaseNetMTBF (90 days)
over 1 year.

\textbf{ Preconditions}:\\
\begin{enumerate}
\tightlist
\item
  PMCS DMTC-7400-2400 Complete.
\item
  6 months of historical availability data for this link is available.
\item
  perSonar installed in Summit and publishing statistics to MadDash.
\item
  As-built documentation for all of the above is available.
\end{enumerate}

NOTE: After the initial test, the corresponding verification elements
will be flagged as ``Requires Monitoring'' such that those requirements
will be closed out as having been verified but will continue to be
monitored throughout commissioning to ensure they do not drop out of
compliance. This will also be monitored for end to end Summit - Data
Facility transfers during Commissioning.

Final comment:\\


Detailed steps :

\begin{tabular}{p{2cm}}
\toprule
Step 1  \\ \hline
\end{tabular}
 Description \\
{\footnotesize
Monitor summit to base networking for at least 1 week

}
\hdashrule[0.5ex]{\textwidth}{1pt}{3mm}
  Test Data \\
 {\footnotesize
LATISS, ComCAM, and/or Full Camera data.

}
\hdashrule[0.5ex]{\textwidth}{1pt}{3mm}
  Expected Result \\
{\footnotesize
Summit - base network is operational for 1 week and monitoring data is
collected.

}

\begin{tabular}{p{2cm}}
\toprule
Step 2  \\ \hline
\end{tabular}
 Description \\
{\footnotesize
Extrapolate annual availability, compare with at least 6 months of
historical data on the link.

}
\hdashrule[0.5ex]{\textwidth}{1pt}{3mm}
  Test Data \\
 {\footnotesize
Historical and current logs

}
\hdashrule[0.5ex]{\textwidth}{1pt}{3mm}
  Expected Result \\
{\footnotesize
The mean time between failures (MTBF) is projected to be less than
summToBaseNetMTBF (90 days) over 1 year.

}

\paragraph{ LVV-T188 - Verify implementation of Summit to Base Network Ownership and Operation }\mbox{}\\

Version \textbf{1}.
Open  \href{https://jira.lsstcorp.org/secure/Tests.jspa#/testCase/LVV-T188}{\textit{ LVV-T188 } }
test case in Jira.

Verify Summit to Base Network Ownership and Operation by LSST and/or the
operations entity by inspection of construction and operations contracts
and Indefeasible Rights.

\textbf{ Preconditions}:\\
\begin{enumerate}
\tightlist
\item
  As-built documentation for all of the above contracts and IRUs is
  available.
\end{enumerate}

Final comment:\\


Detailed steps :

\begin{tabular}{p{2cm}}
\toprule
Step 1  \\ \hline
\end{tabular}
 Description \\
{\footnotesize
Examine contracts with REUNA and telefonica for fiber ownership and
maintenance terms.

}
\hdashrule[0.5ex]{\textwidth}{1pt}{3mm}
  Expected Result \\
{\footnotesize
Rubin Observatory is owner of fibers on AURA property and Summit - Base
DWDM ~and has 15-year IRU for use of fibers on all segments. ~REUNA is
owner of LS - SCL DWDM on AURA property and in Santiago, and is operator
on all fibers and DWDM. ~Telefonica is contracted to maintain fibers not
on AURA property.

}

\paragraph{ LVV-T186 - Verify implementation of Summit to Base Network Reliability }\mbox{}\\

Version \textbf{1}.
Open  \href{https://jira.lsstcorp.org/secure/Tests.jspa#/testCase/LVV-T186}{\textit{ LVV-T186 } }
test case in Jira.

Verify the reliability of the summit to base network by demonstrating
reconnection and recovery to transfer of data at or exceeding rates
specified in \citeds{LDM-142} following a cut in network connection, within MTTR
specification. The network operator will provide MTTR data on links
during commissioning and operations.\\[2\baselineskip]

\textbf{ Preconditions}:\\
\begin{enumerate}
\tightlist
\item
  PMCS DMTC-7400-2400 Complete
\item
  As-built documentation for Summit - Base Network is available.
\end{enumerate}

NOTE: After the initial test, the corresponding verification elements
will be flagged as ``Requires Monitoring'' such that those requirements
will be closed out as having been verified but will continue to be
monitored throughout commissioning to ensure they do not drop out of
compliance. This will also be monitored for end to end Summit - Data
Facility transfers during Commissioning.

Final comment:\\


Detailed steps :

\begin{tabular}{p{2cm}}
\toprule
Step 1  \\ \hline
\end{tabular}
 Description \\
{\footnotesize
Disconnect fiber cable at an endpoint location on the base side of the
Summit - Base fiber.

}
\hdashrule[0.5ex]{\textwidth}{1pt}{3mm}
  Test Data \\
 {\footnotesize
\begin{itemize}
\tightlist
\item
  LATISS, ComCAM, or LSSTCam data
\end{itemize}

}
\hdashrule[0.5ex]{\textwidth}{1pt}{3mm}
  Expected Result \\
{\footnotesize
Fiber is disconnected and the fault is detected by the network
monitoring system.

}

\begin{tabular}{p{2cm}}
\toprule
Step 2  \\ \hline
\end{tabular}
 Description \\
{\footnotesize
Measure the cable with the OTDR to locate the distance from the end
point. Diagnose that it is a break.

}
\hdashrule[0.5ex]{\textwidth}{1pt}{3mm}
  Test Data \\
 {\footnotesize
NA

}
\hdashrule[0.5ex]{\textwidth}{1pt}{3mm}
  Expected Result \\
{\footnotesize
OTDR shows the fiber is disconnected (break).

}

\begin{tabular}{p{2cm}}
\toprule
Step 3  \\ \hline
\end{tabular}
 Description \\
{\footnotesize
Elapse time to simulate the following:

\begin{itemize}
\tightlist
\item
  Go to the most inaccessible place which would mean carrying all the
  tools/splicer/generator/tent equipment some ~metres.
\item
  Erect a tent to make the splice
\item
  Start the generator
\item
  Do a splice on some random piece of cable
\item
  At an end point measure the cable again to ensure it is break free.
\item
  Take down and reinstall an isolated pole (not in the actual fiber
  path)
\item
  Put the cable on the pole.
\end{itemize}

}
\hdashrule[0.5ex]{\textwidth}{1pt}{3mm}
  Test Data \\
 {\footnotesize
NA

}
\hdashrule[0.5ex]{\textwidth}{1pt}{3mm}
  Expected Result \\
{\footnotesize
Wall clock advances by 24 hours.

}

\begin{tabular}{p{2cm}}
\toprule
Step 4  \\ \hline
\end{tabular}
 Description \\
{\footnotesize
Clean fiber connections. ~Restore connection (e.g. reconnect cable).
~Cycle equipment as necessary to confirm fiber is connected.

}
\hdashrule[0.5ex]{\textwidth}{1pt}{3mm}
  Test Data \\
 {\footnotesize
NA

}
\hdashrule[0.5ex]{\textwidth}{1pt}{3mm}
  Expected Result \\
{\footnotesize
Network recovers and resumes sending data.

}

\begin{tabular}{p{2cm}}
\toprule
Step 5  \\ \hline
\end{tabular}
 Description \\
{\footnotesize
Measure with OTDR to ensure back to normal state.

}
\hdashrule[0.5ex]{\textwidth}{1pt}{3mm}
  Test Data \\
 {\footnotesize
NA

}
\hdashrule[0.5ex]{\textwidth}{1pt}{3mm}
  Expected Result \\
{\footnotesize
OTDR indicates normal state.

}

\paragraph{ LVV-T187 - Verify implementation of Summit to Base Network Secondary Link }\mbox{}\\

Version \textbf{1}.
Open  \href{https://jira.lsstcorp.org/secure/Tests.jspa#/testCase/LVV-T187}{\textit{ LVV-T187 } }
test case in Jira.

Verify automated fail-over from primary to secondary equipment in Rubin
Observatory DWDM on simulated failure of primary. ~Verify bandwidth
sufficiency on secondary. ~Verify automated recovery to primary
equipment on simulated restoration of primary. ~Repeat for failure of
Rubin Observatory fiber and fail-over to AURA fiber and DWDM.
~Demonstrate use of secondary in ``catch-up'' mode.

\textbf{ Preconditions}:\\
\begin{enumerate}
\tightlist
\item
  PMCS DMTC-7400-2400 complete.
\item
  As-built documentation for Summit - Base Network is available.
\end{enumerate}

NOTE: After the initial test, the corresponding verification elements
will be flagged as ``Requires Monitoring'' such that those requirements
will be closed out as having been verified but will continue to be
monitored throughout commissioning to ensure they do not drop out of
compliance. This will also be monitored for end to end Summit - Data
Facility transfers during Commissioning.

Final comment:\\


Detailed steps :

\begin{tabular}{p{2cm}}
\toprule
Step 1  \\ \hline
\end{tabular}
 Description \\
{\footnotesize
Transfer data between summit and base on primary equipment (LSST Summit
- Base) over uninterrupted 1 day period. ~

}
\hdashrule[0.5ex]{\textwidth}{1pt}{3mm}
  Test Data \\
 {\footnotesize
LATISS, ComCAM, or LSSTCAM data.

}
\hdashrule[0.5ex]{\textwidth}{1pt}{3mm}
  Expected Result \\
{\footnotesize
Normal operations.

}

\begin{tabular}{p{2cm}}
\toprule
Step 2  \\ \hline
\end{tabular}
 Description \\
{\footnotesize
Simulate equipment outage by disconnecting power card from primary DWDM
equipment on base side of Summit - Base Fiber.

}
\hdashrule[0.5ex]{\textwidth}{1pt}{3mm}
  Test Data \\
 {\footnotesize
NA

}
\hdashrule[0.5ex]{\textwidth}{1pt}{3mm}
  Expected Result \\
{\footnotesize
Network fails over to secondary equipment in \textless{}=60s.

}

\begin{tabular}{p{2cm}}
\toprule
Step 3  \\ \hline
\end{tabular}
 Description \\
{\footnotesize
Transfer data between summit and base over secondary equipment
uninterrupted 1 day period while monitoring network.

}
\hdashrule[0.5ex]{\textwidth}{1pt}{3mm}
  Test Data \\
 {\footnotesize
NA

}
\hdashrule[0.5ex]{\textwidth}{1pt}{3mm}
  Expected Result \\
{\footnotesize
Verify that secondary equipment is capable of transferring 1 night of
raw data (nCalibExpDay + nRawExpNightMax = 450 + 2800 = ~3250 exposures)
within summToBaseNet2TransMax (72 hours), i.e. at or exceeding rates
specified in LDM-142.

}

\begin{tabular}{p{2cm}}
\toprule
Step 4  \\ \hline
\end{tabular}
 Description \\
{\footnotesize
Restore ~primary equipment (i.e. reconnect power card to primary
equipment.)

}
\hdashrule[0.5ex]{\textwidth}{1pt}{3mm}
  Test Data \\
 {\footnotesize
NA

}
\hdashrule[0.5ex]{\textwidth}{1pt}{3mm}
  Expected Result \\
{\footnotesize
Network recovers to primary in \textless{}= 60s.

}

\begin{tabular}{p{2cm}}
\toprule
Step 5  \\ \hline
\end{tabular}
 Description \\
{\footnotesize
Simulate fiber outage by disconnecting fiber from primary DWDM equipment
on base side of Summit - Base Fiber.

}
\hdashrule[0.5ex]{\textwidth}{1pt}{3mm}
  Test Data \\
 {\footnotesize
NA

}
\hdashrule[0.5ex]{\textwidth}{1pt}{3mm}
  Expected Result \\
{\footnotesize
Network fails over to AURA DWDM and fiber.

}

\begin{tabular}{p{2cm}}
\toprule
Step 6  \\ \hline
\end{tabular}
 Description \\
{\footnotesize
Transfer data between summit and base over AURA fiber and equipment
uninterrupted 1 day period while monitoring network.

}
\hdashrule[0.5ex]{\textwidth}{1pt}{3mm}
  Test Data \\
 {\footnotesize
LATISS, ComCAM, or FullCAM data.

}
\hdashrule[0.5ex]{\textwidth}{1pt}{3mm}
  Expected Result \\
{\footnotesize
Verify that AURA fiber and equipment is capable of transferring 1 night
of raw data (nCalibExpDay + nRawExpNightMax = 450 + 2800 = ~3250
exposures) within summToBaseNet2TransMax (72 hours), i.e. at or
exceeding rates specified in LDM-142.

}

\begin{tabular}{p{2cm}}
\toprule
Step 7  \\ \hline
\end{tabular}
 Description \\
{\footnotesize
Restore ~primary fiber (i.e. reconnect fiber to Rubin Observatory DWDM
equipment.)

}
\hdashrule[0.5ex]{\textwidth}{1pt}{3mm}
  Expected Result \\
{\footnotesize
Network recovers to Rubin Observatory fiber and DWDM.

}

\begin{tabular}{p{2cm}}
\toprule
Step 8  \\ \hline
\end{tabular}
 Description \\
{\footnotesize
Demonstrate use of secondary in ``catch-up'' mode.

}
\hdashrule[0.5ex]{\textwidth}{1pt}{3mm}
  Test Data \\
 {\footnotesize
DAQ data buffer full of images and associated meta-data

}
\hdashrule[0.5ex]{\textwidth}{1pt}{3mm}
  Expected Result \\
{\footnotesize
Images from DAQ buffer and associated metadata are retrievable over
secondary path while current observing data is being transferred over
primary path.

}

\paragraph{ LVV-T193 - Verify implementation of Base to Archive Network }\mbox{}\\

Version \textbf{1}.
Open  \href{https://jira.lsstcorp.org/secure/Tests.jspa#/testCase/LVV-T193}{\textit{ LVV-T193 } }
test case in Jira.

Verify that the data acquired by a DAQ can be transferred within the
required time, i.e. verify that link is capable of transferring image
for prompt processing in oArchiveMaxTransferTime = 5{[}second{]}, i.e.
at or exceeding rates specified in \citeds{LDM-142}.

\textbf{ Preconditions}:\\
\begin{enumerate}
\tightlist
\item
  Archiver/Forwarders are configured at Base, connected to REUNA DWDM,
  loaded with simulated or pre-cursor data, running on end node
  computers that are the production hardware or equivalent to it.
\item
  Archiver/Forwarder receivers or other capability is on configured at
  LDF, connected to Base - Archive Network, running on end node
  computers that are the production hardware or equivalent to it.
\item
  As-built documentation for all of the above is available.
\end{enumerate}

NOTE: This test will be repeated at increasing data volumes as
additional observatory capabilities (e.g. ComCAM, FullCam) become
available. ~Final verification will be tested at full operational
volume. After the initial test, the corresponding verification elements
will be flagged as ``Requires Monitoring'' such that those requirements
will be closed out as having been verified but will continue to be
monitored throughout commissioning to ensure they do not drop out of
compliance. This will also be monitored for end to end Summit - Data
Facility transfers during Commissioning.

Final comment:\\


Detailed steps :

\begin{tabular}{p{2cm}}
\toprule
Step 1  \\ \hline
\end{tabular}
 Description \\
{\footnotesize
Transfer data between base and archive while monitoring the network over
uninterrupted 1 day period (with repeated transfers on normal observing
cadence).

}
\hdashrule[0.5ex]{\textwidth}{1pt}{3mm}
  Test Data \\
 {\footnotesize
LATISS, ComCAM, or FullCAM data.

}
\hdashrule[0.5ex]{\textwidth}{1pt}{3mm}
  Expected Result \\
{\footnotesize
Data transfers occur without significant delay or frequent latency
spikes.

}

\begin{tabular}{p{2cm}}
\toprule
Step 2  \\ \hline
\end{tabular}
 Description \\
{\footnotesize
~Analyze the network logs and monitoring system to determine average and
peak latency and packet loss statistics.

}
\hdashrule[0.5ex]{\textwidth}{1pt}{3mm}
  Expected Result \\
{\footnotesize
Data can be transferred within the required time, i.e. verify that link
is capable of transferring image for prompt processing in
oArchiveMaxTransferTime = 5{[}second{]}. Verify transfer of data at or
exceeding rates specified in LDM-142 at least 98\% of the time.

}

\paragraph{ LVV-T194 - Verify implementation of Base to Archive Network Availability }\mbox{}\\

Version \textbf{1}.
Open  \href{https://jira.lsstcorp.org/secure/Tests.jspa#/testCase/LVV-T194}{\textit{ LVV-T194 } }
test case in Jira.

Verify the availability of the Base to Archive Network communications by
demonstrating that it meets or exceeds a mean time between failures,
measured over a 1-yr period of MTBF \textgreater{} baseToArchNetMTBF
(180{[}day{]})

\textbf{ Preconditions}:\\
\begin{enumerate}
\tightlist
\item
  Archiver/Forwarders are configured at Base, connected to REUNA DWDM,
  loaded with simulated or pre-cursor data, running on end node
  computers that are the production hardware or equivalent to it.
\item
  Archiver/Forwarder receivers or other capability is on configured at
  LDF, connected to Base - Archive Network, running on end node
  computers that are the production hardware or equivalent to it.
\item
  At least 6 months of historical monitoring data on this link is
  available.
\item
  As-built documentation for all of the above is available.
\end{enumerate}

NOTE: This test will be repeated at increasing data volumes as
additional observatory capabilities (e.g. ComCAM, FullCam) become
available. Final verification will be tested at full operational volume.
After the initial test, the corresponding verification elements will be
flagged as ``Requires Monitoring'' such that those requirements will be
closed out as having been verified but will continue to be monitored
throughout commissioning to ensure they do not drop out of compliance.
This will also be monitored for end to end Summit - Data Facility
transfers during Commissioning.

Final comment:\\


Detailed steps :

\begin{tabular}{p{2cm}}
\toprule
Step 1  \\ \hline
\end{tabular}
 Description \\
{\footnotesize
Transfer data between base and archive over uninterrupted 1 week period.

}
\hdashrule[0.5ex]{\textwidth}{1pt}{3mm}
  Test Data \\
 {\footnotesize
LATISS, ComCAM, or FullCAM data.

}
\hdashrule[0.5ex]{\textwidth}{1pt}{3mm}
  Expected Result \\
{\footnotesize
Data is successfully transferred during the entire week.

}

\begin{tabular}{p{2cm}}
\toprule
Step 2  \\ \hline
\end{tabular}
 Description \\
{\footnotesize
Analyze monitoring/performance data, compare to historical data, and
extrapolate to a full year, average and peak throughput and latency.

}
\hdashrule[0.5ex]{\textwidth}{1pt}{3mm}
  Test Data \\
 {\footnotesize
NA

}
\hdashrule[0.5ex]{\textwidth}{1pt}{3mm}
  Expected Result \\
{\footnotesize
Extrapolated network availability meets baseToArchNetMTBF =
180{[}day{]}. ~Note that this is for complete loss of transfer service
(all paths), not a single path failure with successful fail-over.

}

\paragraph{ LVV-T195 - Verify implementation of Base to Archive Network Reliability }\mbox{}\\

Version \textbf{1}.
Open  \href{https://jira.lsstcorp.org/secure/Tests.jspa#/testCase/LVV-T195}{\textit{ LVV-T195 } }
test case in Jira.

Verify Base to Archive Network Reliability by demonstrating that the
network can recover from outages within baseToArchNetMTTR =
48{[}hour{]}.

\textbf{ Preconditions}:\\
\begin{enumerate}
\tightlist
\item
  Archiver/Forwarders are configured at Base, connected to REUNA DWDM,
  loaded with simulated or pre-cursor data, running on end node
  computers that are the production hardware or equivalent to it.
\item
  Archiver/Forwarder receivers or other capability is on configured at
  LDF, connected to Base - Archive Network, running on end node
  computers that are the production hardware or equivalent to it.
\item
  At least 6 months of monitoring data for this link is available.
\item
  As-built documentation for all of the above is available.
\end{enumerate}

NOTE: This test will be repeated at increasing data volumes as
additional observatory capabilities (e.g. ComCAM, FullCam) become
available. Final verification will be tested at full operational volume.
After the initial test, the corresponding verification elements will be
flagged as ``Requires Monitoring'' such that those requirements will be
closed out as having been verified but will continue to be monitored
throughout commissioning to ensure they do not drop out of compliance.
This will also be monitored for end to end Summit - Data Facility
transfers during Commissioning.\\[2\baselineskip]

Final comment:\\


Detailed steps :

\begin{tabular}{p{2cm}}
\toprule
Step 1  \\ \hline
\end{tabular}
 Description \\
{\footnotesize
Disconnect primary fiber on base side of Base - ~Archive network.

}
\hdashrule[0.5ex]{\textwidth}{1pt}{3mm}
  Test Data \\
 {\footnotesize
LATISS, ComCAM, or FullCAM data.

}
\hdashrule[0.5ex]{\textwidth}{1pt}{3mm}
  Expected Result \\
{\footnotesize
Network fails over to secondary path.

}

\begin{tabular}{p{2cm}}
\toprule
Step 2  \\ \hline
\end{tabular}
 Description \\
{\footnotesize
Simulate diagnosis and repair by elapsed time.\\[2\baselineskip]

}
\hdashrule[0.5ex]{\textwidth}{1pt}{3mm}
  Test Data \\
 {\footnotesize
NA

}
\hdashrule[0.5ex]{\textwidth}{1pt}{3mm}
  Expected Result \\
{\footnotesize
Wall clock advances by 48 hours. ~Data is successfully transferred over
secondary path.

}

\begin{tabular}{p{2cm}}
\toprule
Step 3  \\ \hline
\end{tabular}
 Description \\
{\footnotesize
Reconnect primary fiber on base side of Base - Archive network.

}
\hdashrule[0.5ex]{\textwidth}{1pt}{3mm}
  Test Data \\
 {\footnotesize
NA

}
\hdashrule[0.5ex]{\textwidth}{1pt}{3mm}
  Expected Result \\
{\footnotesize
Network recovers to primary path.~

}

\begin{tabular}{p{2cm}}
\toprule
Step 4  \\ \hline
\end{tabular}
 Description \\
{\footnotesize
Analyze fail-over and recovery times. ~Compare to historical data and
extrapolate to MTTR.

}
\hdashrule[0.5ex]{\textwidth}{1pt}{3mm}
  Expected Result \\
{\footnotesize
Verify recovery can occur within baseToArchNetMTTR = 48{[}hour{]}.
Demonstrate reconnection and recovery to transfer of data at or
exceeding rates specified in LDM-142.

}

\paragraph{ LVV-T196 - Verify implementation of Base to Archive Network Secondary Link }\mbox{}\\

Version \textbf{1}.
Open  \href{https://jira.lsstcorp.org/secure/Tests.jspa#/testCase/LVV-T196}{\textit{ LVV-T196 } }
test case in Jira.

Verify Base to Archive Network Secondary Link failover and capacity, and
subsequent recovery primary. Demonstrate the use of the secondary path
in ``catch-up'' mode.

\textbf{ Preconditions}:\\
\begin{enumerate}
\tightlist
\item
  Archiver/Forwarders are configured at Base, connected to REUNA DWDM,
  loaded with simulated or pre-cursor data, running on end node
  computers that are the production hardware or equivalent to it.
\item
  Archiver/Forwarder receivers or other capability is on configured at
  LDF, connected to Base - Archive Network, running on end node
  computers that are the production hardware or equivalent to it.
\item
  As-built documentation for all of the above is available.
\end{enumerate}

NOTE: This test will be repeated at increasing data volumes as
additional observatory capabilities (e.g. ComCAM, FullCam) become
available. Final verification will be tested at full operational volume.
After the initial test, the corresponding verification elements will be
flagged as ``Requires Monitoring'' such that those requirements will be
closed out as having been verified but will continue to be monitored
throughout commissioning to ensure they do not drop out of compliance.
This will also be monitored for end to end Summit - Data Facility
transfers during Commissioning.\\[2\baselineskip]

Final comment:\\


Detailed steps :

\begin{tabular}{p{2cm}}
\toprule
Step 1  \\ \hline
\end{tabular}
 Description \\
{\footnotesize
Transfer data between base and archive on primary links over
uninterrupted 1 day period.

}
\hdashrule[0.5ex]{\textwidth}{1pt}{3mm}
  Test Data \\
 {\footnotesize
LATISS, ComCAM, or FullCAM data.

}
\hdashrule[0.5ex]{\textwidth}{1pt}{3mm}
  Expected Result \\
{\footnotesize
Data is successfully transferred over primary link at or exceeding rates
specified in LDM-142 throughout period.

}

\begin{tabular}{p{2cm}}
\toprule
Step 2  \\ \hline
\end{tabular}
 Description \\
{\footnotesize
Simulate outage by disconnecting fiber on primary fiber on Base side of
Base - Archive Network.

}
\hdashrule[0.5ex]{\textwidth}{1pt}{3mm}
  Test Data \\
 {\footnotesize
NA

}
\hdashrule[0.5ex]{\textwidth}{1pt}{3mm}
  Expected Result \\
{\footnotesize
Network fails over to secondary links in \textless{}=60s

}

\begin{tabular}{p{2cm}}
\toprule
Step 3  \\ \hline
\end{tabular}
 Description \\
{\footnotesize
Transfer data between base and archive over secondary equipment
uninterrupted 1 day period.

}
\hdashrule[0.5ex]{\textwidth}{1pt}{3mm}
  Test Data \\
 {\footnotesize
LATISS, ComCAM, or FullCAM data.

}
\hdashrule[0.5ex]{\textwidth}{1pt}{3mm}
  Expected Result \\
{\footnotesize
Data is successfully transferred over secondary link ~at or exceeding
rates specified in LDM-142 throughout period.

}

\begin{tabular}{p{2cm}}
\toprule
Step 4  \\ \hline
\end{tabular}
 Description \\
{\footnotesize
Restore connection on primary link by reconnecting
fiber.\\[2\baselineskip]

}
\hdashrule[0.5ex]{\textwidth}{1pt}{3mm}
  Test Data \\
 {\footnotesize
NA

}
\hdashrule[0.5ex]{\textwidth}{1pt}{3mm}
  Expected Result \\
{\footnotesize
Network recovers to primary.

}

\begin{tabular}{p{2cm}}
\toprule
Step 5  \\ \hline
\end{tabular}
 Description \\
{\footnotesize
Demonstrate use of secondary in catch-up mode.

}
\hdashrule[0.5ex]{\textwidth}{1pt}{3mm}
  Test Data \\
 {\footnotesize
DAQ buffer full of images and associated metadata.

}
\hdashrule[0.5ex]{\textwidth}{1pt}{3mm}
  Expected Result \\
{\footnotesize
Images from DAQ buffer and associated metadata are retrievable over
secondary path while current observing data is being transferred over
primary path.

}




% This appendix is put in as part of the template. You may edit and add to it.
% It is not overwritten by Docsteady.

\newpage
\appendix
\section{Documentation}
The verification process is defined in \citeds{LSE-160}.
The use of Docsteady to format Jira information in various test and planing documents is
described in \citeds{DMTN-140} and practical commands are given in \citeds{DMTN-178}.

\section{Acronyms used in this document}\label{sec:acronyms}
\input{acronyms.tex}

\newpage

% Uncomment this if Docsteady makes you additional appendix
%% generated from JIRA project LVV
% using template at /usr/local/lib/python3.7/site-packages/docsteady/templates/dm-tpr-appendix.latex.jinja2.
% using docsteady version 1.2rc24
% Please do not edit -- update information in Jira instead

\section{Traceability}

\begin{longtable}{p{3cm}p{3cm}L{9cm}}
\hline
\textbf{Test Case} & \textbf{VE Key} & \textbf{VE Summary} \\ \hline
\href{https://jira.lsstcorp.org/secure/Tests.jspa#/testCase/LVV-T193}{LVV-T193} &
  \href{https://jira.lsstcorp.org/browse/LVV-81}{LVV-81}
  & DMS-REQ-0180-V-01: Base to Archive Network
 \\ \cdashline{2-3}
\hline
\href{https://jira.lsstcorp.org/secure/Tests.jspa#/testCase/LVV-T194}{LVV-T194} &
  \href{https://jira.lsstcorp.org/browse/LVV-82}{LVV-82}
  & DMS-REQ-0181-V-01: Base to Archive Network Availability
 \\ \cdashline{2-3}
\hline
\href{https://jira.lsstcorp.org/secure/Tests.jspa#/testCase/LVV-T195}{LVV-T195} &
  \href{https://jira.lsstcorp.org/browse/LVV-83}{LVV-83}
  & DMS-REQ-0182-V-01: Base to Archive Network Reliability
 \\ \cdashline{2-3}
\hline
\href{https://jira.lsstcorp.org/secure/Tests.jspa#/testCase/LVV-T196}{LVV-T196} &
  \href{https://jira.lsstcorp.org/browse/LVV-84}{LVV-84}
  & DMS-REQ-0183-V-01: Base to Archive Network Secondary Link
 \\ \cdashline{2-3}
\hline
\end{longtable}


\end{document}
